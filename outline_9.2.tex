\documentclass[11pt]{article}
\usepackage[letterpaper, margin=1in]{geometry}
\usepackage{amsmath, amssymb, graphicx, epsfig, fleqn}
\setlength{\parindent}{0pt}
\newcommand{\ud}{\,\mathrm{d}}
\everymath{\displaystyle}
\def\FillInBlank{\rule{2.5in}{.01in} }
\pagestyle{empty}

\begin{document}
\begin{center}
\Large
\rm{Math 112}
\\
\rm{Chapter 9.2:  Euler's Method}\\

\end{center}
\vspace{0.2in}
\fboxsep0.5cm

\vspace{0.2in}

The goal of {\bf Euler's Method} is to approximate the solution of a differential equation at
at a discrete set of points.  \\

IDEA:  We can follow the direction field for a short time in order to appoximate the solution.

\vspace{6in}

    {\bf Euler's Method:}

\begin{displaymath}
  y_n = y_{n-1} + hf(x_{n-1},y_{n-1})\quad \mbox{where}\quad x_n=x_0+nh
\end{displaymath}

\pagebreak
EXAMPLES:\\

Use Euler's Method with $h=0.1$ to approximate $y(0.3)$ where $y(x)$ is a solution to the
initial value problem

\begin{displaymath}
  \left\{ \begin{array}{ll}
  y' = y + x\\
y(0) = 1 \\
\end{array} \right.
\end{displaymath}

\vspace{3.25in}

Use Euler's Method with $h=0.02$ to approximate $y(1.1)$ where $y(x)$ is a solution to the
initial value problem

\begin{displaymath}
  \left\{ \begin{array}{ll}
  y' = y\cos(x)\\
y(1) = 2 \\
\end{array} \right.
\end{displaymath}

\vspace{0.8in}
\pagebreak
\begin{center}
\Large
\rm{Chapter 9.3:  Separable Differential Equations}\\

\end{center}

A differential equation is called {\bf separable} if can be written in the form

\begin{displaymath}
  \frac{dy}{dx} = g(x)f(y)
\end{displaymath}

\vspace{0.2in}
Solutions can be found by ``separating'' the variables and then integrating.\\

\vspace{2.5in}

EXAMPLES:

\begin{displaymath}
  \frac{dy}{dx} = ky
\end{displaymath}

\vspace{2in}

\begin{displaymath}
  \frac{dy}{dx} = 3x^2y
\end{displaymath}

\pagebreak

Solve the initial value problems:\\

\begin{displaymath}
  \left\{ \begin{array}{ll}
  \frac{dy}{dt} = y^2\sin{t} \\
y(0) = 4 \\
\end{array} \right.
\end{displaymath}

\vspace{2in}

\begin{displaymath}
  \left\{ \begin{array}{ll}
  \frac{dy}{dt} = 3-4y\\
y(0) = 10 \\
\end{array} \right.
\end{displaymath}

\vspace{2.5in}


\begin{displaymath}
  \left\{ \begin{array}{ll}
  \frac{dy}{dx} = \frac{-x}{y} \\
y(1) = -3 \\
\end{array} \right.
\end{displaymath}


\vspace{2in}

NOTE:  It may not always be possible to ``solve'' explicity for $y(x)$.

\begin{displaymath}
  \frac{dy}{dx} = \frac{x^2}{1-y^2}
\end{displaymath}


\vspace{3.5in}

APPLICATIONS:\\

Solve the initial value problem for the velocity of a falling object that is subject to a drag force

\begin{displaymath}
  \left\{ \begin{array}{ll}
  m\frac{dv}{dt} = kv-mg\\
v(0) = v_0 \\
\end{array} \right.
\end{displaymath}

\pagebreak


A tank contains 200 L of water with 0.8 kg of salt.   At $t=0$, a solution with concentration 3.5 g/L begins
flowing into the tank at a rate of 2 L/min. If the tank is kept well-mixed and is drained at a rate of 2 L/min,
find the amount of salt in the tank as a function of time.


\vspace{4.5in}

Find the family of curves that is orthogonal to every member of the family $y=\frac{k}{x}$.
%
%\pagebreak
%
%Let $h(t)$ and $V(t)$ be the height and volume of water in a tank at time $t$.  If the tank drains through a hole
%at the bottom with radius $a$, then Torrecelli's Law is the following equation.
%
%\begin{displaymath}
%  \frac{dV}{dt} = -a\sqrt{2gh}
%\end{displaymath}
%
%\vspace{1in}


\end{document}




