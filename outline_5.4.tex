\documentclass[11pt]{article}
\usepackage[letterpaper, margin=1in]{geometry}
\usepackage{amsmath, amssymb, graphicx, epsfig, fleqn}
\setlength{\parindent}{0pt}
\newcommand{\ud}{\,\mathrm{d}}
\everymath{\displaystyle}
\def\FillInBlank{\rule{2.5in}{.01in} }
\pagestyle{empty}

\begin{document}
\begin{center}
\Large
\rm{Math 112}
\\
\rm{Chapter 5.4:  Indefinite Integrals and Net Change}
\\
\end{center}
\vspace{0.2in}
\fboxsep0.5cm

{\bf Indefinite integrals} are a notation for antiderivatives.


\vspace{3in}


NOTES:  \\

\begin{enumerate}

\item{Definte integrals are numbers, indefinite integrals are functions.}

  \vspace{1in.}

\item{Fundamental Theorem connects definite and indefinite integrals.}

  \vspace{1in.}
  
\item{Indefinite integrals represent a family of functions.}

    \vspace{1in.}

    \end{enumerate}


\pagebreak

    EXAMPLES: \\
    
 $\int \sec^2{\theta} \, d\theta$

    \vspace{1in}

   
$\int \frac{2}{x}+  \frac{x}{8}\, dx$
    
    
    \vspace{1in}

        $\int \frac{\sin{t}}{\cos^2{t}} \, dt$

    \vspace{1in}

{\bf Net Change Theorem} (FTC Part II)\\

The integral of a rate of change equals the net change.

\begin{displaymath}
\int_a^bF'(x) = F(b)-F(a)
\end{displaymath}

\vspace{.5in}

EXAMPLES:\\

\begin{enumerate}
\item{Let $V(t)$ be the volume of water in a tank as a function of time.  Then $\int_a^b\frac{dV}{dt} \,dt = \Delta V$.}

\pagebreak

\item{A colony of bactera starts at population 100 and grows at rate of $200e^t$ bactera/hr.  What is the population after 4 hours?}

  \vspace{2.5in.}

  
\item{ If $x(t)$ represents the position of an object, and $v(t) = x'(t)$ its velocity, then $\int_a^bv(t)\, dt$ calculates the displacement from time $a$ to time $b$.  Suppose $v(t)=3t^2-24t+36$  m/s.
  \begin{enumerate}
  \item{What is the displacement during the interval $[0,2]$?}
    
    \vspace{1.2in.}

  \item{What is the displacement during the interval $[0,6]$?}

    \vspace{1.2in.}

  \item{What is the distance traveled during the interval $[0,6]$?}
\end{enumerate}
}
\end{enumerate}


\pagebreak
\begin{center}
\Large
\rm{Chapter 5.5:  The Substitution Rule}
\\
\end{center}
\vspace{0.2in}
In order to compute antiderivatives, we will need to use differentiation rules in reverse.\\

Chain Rule review:  Differentiate the following functions

\begin{displaymath}
h(x) = (x^2+4x^4)^{10}  \hspace{1in} p(t) = \ln{(\tan{t})}  \hspace{1.in} y(u) = \arctan{e^u} 
  \end{displaymath}

\vspace{2.5in}




Evaluate the integral $\int 2x\sqrt{1+x^2}\, dx $.

\vspace{3in}


{\bf Substitution Rule} \\

\begin{displaymath}
\int f(g(x))g'(x) \, dx = \int f(u) \, du
  \end{displaymath}




\pagebreak

MANY EXAMPLES:\\


$\int x^2e^{x^3}\, dx$

\vspace{1.5in}

$\int \frac{x}{\sqrt{1-4x^2}}\, dx$

\vspace{1.5in}

$\int e^{2x}\cos{e^{2x}} \, dx$

\vspace{1.5in}

$\int (3w+1)^3 \, dw$


\vspace{1.5in}

$\int \tan{\theta} \, d\theta$

\pagebreak


$\int_0^4 \sqrt{4+3x} \, dx$

\vspace{2in}

$\int_0^{\sqrt{\pi}} t\sin{t^2} \, dt$

\vspace{1.5in}

$\int_0^{\ln{3}} \frac{e^x}{1+e^x} \, dx$

\vspace{1.5in}


$\int_1^{e^{2}} \frac{\ln{x}}{x} \, dx$

\vspace{1.5in}

$\int_3^{4} \frac{dx}{(x-2)^2} $


\end{document}


