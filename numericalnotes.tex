\documentclass[11pt]{article}
\usepackage[margin=0.75in]{geometry}
\usepackage{amsmath, amssymb, graphicx, psfrag, epsfig, fleqn}
\pagestyle{empty}
\setlength\parindent{0pt}
\begin{document}

%\maketitle
\begin{center}
\Large
\rm{Numerical Integration}
\end{center}
\vspace{0.05in}

The following formulas are used to find approximations of definite integrals when the antiderivatives involved are unknown.
In all of the formulas, $N$ is an integer that represents the number of pieces in to which the integral is divided, $\Delta x = (b-a)/N$, 
and $x_i = a+i\Delta x$. 

\vspace{0.5cm}

{\bf Midpoint Rule}
\begin{displaymath}
\int_a^bf(x)\,dx \approx M_N = \Delta x\left[f(\bar{x}_1) + f(\bar{x}_2) + f(\bar{x}_3) + .... + f(\bar{x}_{N-1}) + f(\bar{x}_N)\right]
\end{displaymath}
where $\bar{x}_i = (x_i+x_{i-1})/2$ are the midpoints of the intervals $[x_{i-1},x_i]$.

\vspace{0.5cm}

{\bf Trapezoid Rule}
\begin{displaymath}
\int_a^bf(x)\,dx \approx T_N = \frac{\Delta x}{2}\left[f(x_0) + 2f(x_1) + 2f(x_2) + .... + 2f(x_{N-1}) + f(x_N)\right]
\end{displaymath}

\vspace{0.5cm}

{\bf Simpson's Rule}
\begin{displaymath}
\int_a^bf(x)\,dx \approx S_N = \frac{\Delta x}{3}\left[f(x_0) + 4f(x_1) + 2f(x_2) +  4f(x_3) + .... + 2f(x_{N-2}) + 4f(x_{N-1}) + f(x_N)\right]
\end{displaymath}
where $N$ {\bf is an even integer.}

\vspace{0.5cm}

{\bf Example:}
Using the rules above, and $N=6$, we will approximate the integral 
\begin{displaymath}
\int_1^4\sqrt{\ln{x}}\,dx 
\end{displaymath}
For all three formulas $\Delta x = (4-1)/6 = 0.5$, $x_0=1$, $x_1=1.5$, $x_2=2$, $x_3=2.5$, $x_4=3$, $x_5=3.5$, $x_6=4$.  
\\
For the Midpoint Rule, we calculate $\bar{x}_1=1.25$, $\bar{x}_2=1.75$, $\bar{x}_3=2.25$, $\bar{x}_4=2.75$, $\bar{x}_5=3.25$, $\bar{x}_6=3.75$. 
\begin{eqnarray*}
M_6 &=& 0.5\left[f(1.25) + f(1.75) + f(2.25) + f(2.75) + f(3.25) + f(3.75)\right] \\
    &=& 0.5\left[\sqrt{\ln{1.25}} + \sqrt{\ln{1.75}} + \sqrt{\ln{2.25}} + \sqrt{\ln{2.75}} + \sqrt{\ln{3.25}} + \sqrt{\ln{3.75}}\right] \\
    &\approx & 2.6810
\end{eqnarray*}

For the Trapezoid Rule:
\begin{eqnarray*}
T_6 &=& \frac{0.5}{2}\left[f(1.0) + 2f(1.5) + 2f(2.0) + 2f(2.5) + 2f(3.0) + 2f(3.5)+ f(4.0)\right] \\
    &=& \frac{0.5}{2}\left[\sqrt{\ln{1}} + 2\sqrt{\ln{1.5}} + 2\sqrt{\ln{2.0}} + 2\sqrt{\ln{2.5}} + 2\sqrt{\ln{3.0}} + 2\sqrt{\ln{3.5}} + \sqrt{\ln{4}}\right] \\
    &\approx & 2.5913
\end{eqnarray*}

For Simpson's Rule:
\begin{eqnarray*}
S_6 &=& \frac{0.5}{3}\left[f(1.0) + 4f(1.5) + 2f(2.0) + 4f(2.5) + 2f(3.0) + 4f(3.5)+ f(4.0)\right] \\
    &=& \frac{0.5}{3}\left[\sqrt{\ln{1}} + 4\sqrt{\ln{1.5}} + 2\sqrt{\ln{2.0}} + 4\sqrt{\ln{2.5}} + 2\sqrt{\ln{3.0}} + 4\sqrt{\ln{3.5}} + \sqrt{\ln{4}}\right] \\
    &\approx & 2.6319
\end{eqnarray*}

{\bf Exercise:}
Find $M_8$, $T_8$, and $S_8$ for the following integral:
\begin{displaymath}
\int_0^2 \frac{dx}{1+x^6}
\end{displaymath}
\pagebreak

{\bf Error bounds}

The formulas provide only approximations to the true value of an integral.  Associated with each formula is an error.  The error is the difference
between the true value of the integral and that produced by the formula.
\begin{displaymath}
E_M = \int_a^bf(x)\,dx - M_N \quad\quad E_T = \int_a^bf(x)\,dx - T_N \quad\quad E_S = \int_a^bf(x)\,dx - S_N
\end{displaymath}
For the Midpoint Rule and Trapezoid rule we have the following formula to estimate the error
\begin{displaymath}
|E_M| \leq \frac{K(b-a)^3}{24N^2} \quad\quad |E_T| \leq \frac{K(b-a)^3}{12N^2}
\end{displaymath}
where $K$ is a number such that $|f''(x)| < K$ when $x$ is in the interval $[a,b]$.
\\
For Simpson's Rule we have a similar formula to estimate the error
\begin{displaymath}
|E_S| \leq \frac{K(b-a)^5}{180N^4} 
\end{displaymath}
where $K$ is a number such that $|f''''(x)| < K$ when $x$ is in the interval $[a,b]$.
\vspace{0.2cm}

{\bf Example:}

Suppose we use the Trapezoid Rule with $N=8$ to approximate the value of the integral 
\begin{displaymath}
\int_0^1\cos{x^2} \,dx 
\end{displaymath}
with $T_8 \approx 0.90233$ (check the number for yourself).  We would now like to know how large the error might be
by using the formula for $E_T$.  We will use $b-a = 1$ and $N=8$.  The more difficult part of the problem is to find 
a value of $K$.  Taking two derivatives of $\cos{x^2}$, we find
\begin{displaymath}
f''(x) = -2\left[\sin{x^2}+2x^2\cos{x^2} \right]
\end{displaymath}
We need a number $K$ so that $|f''(x)| \leq K$.  We notice that $|\sin{x^2}| \leq 1$ and $|\cos{x^2}| \leq 1$ for any value of $x$. 
Also, $x^2 \leq 1$ when $x$ is between 0 and 1.  If we put all of this facts together we get 
\begin{displaymath}
|f''(x)| = |-2\left[\sin{x^2}+2x^2\cos{x^2} \right]| \leq 2\left[1+ 2(1)\right] = 6
\end{displaymath}
Using $K=6$, we can use the error formula to determine that $|E_T| \leq 0.0078125$.  

Alternatively, we might try plotting $|f''(x)|$ and observe that it is maximum when $x=1$.  We could then use
\begin{displaymath}
K = |-2\left[\sin{1}+2\cos{1} \right]| \approx 3.8442
\end{displaymath}
and $|E_T| \leq 0.00501$.  The true error in the $T_8$ approximation must be less than this number.  

\vspace{0.3cm}

{\bf Exercise:}
Consider the following integral:
\begin{displaymath}
\int_0^{\pi}\sin{x} \,dx 
\end{displaymath}
\begin{enumerate}
\item{Find $M_{10}$, $T_{10}$, and $S_{10}$, and the corresponding errors $E_M$, $E_T$, and $E_S$.}
\item{Compute the error bounds using the formulas and then compare them with the true errors $E_M$, $E_T$, and $E_S$.}
\item{How large do we have to choose $N$ so that $T_N$ are accurate to within 0.00001?  Use the error bound formula.}
\end{enumerate}

\end{document}





