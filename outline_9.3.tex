\documentclass[11pt]{article}
\usepackage[letterpaper, margin=1in]{geometry}
\usepackage{amsmath, amssymb, graphicx, epsfig}
\setlength{\parindent}{0pt}
\newcommand{\ud}{\,\mathrm{d}}
\everymath{\displaystyle}
\def\FillInBlank{\rule{2.5in}{.01in} }
\pagestyle{empty}

\begin{document}
\begin{center}
\Large
\rm{Math 112}
\\
\rm{Chapter 9.3:  Population Models}\\

\end{center}
\vspace{0.2in}
\fboxsep0.5cm

\vspace{0.2in}

In this section, we consider a dynamic population $P(t)$, and examine differential equations
that might be used to describe how the population changes with time.

\begin{displaymath}
\frac{dP}{dt} = rP 
\end{displaymath}

Where $r$ is the \emph{per capita growth rate}.   If $r$ is constant, the population grows exponentially,
but we want to consider also what happens if $r$ is not constant.\\

EXAMPLE:  Seasonal growth $r(t) = 0.05\cos(t)$\\

\vspace{2.5in}



EXAMPLE:  Diminishing resources $r(t) = t^{-k}$\\

$\frac{dP}{dt} = \frac{P}{t} $

\pagebreak

$\frac{dP}{dt} = \frac{P}{t^2}$

\vspace{2.5in}

It is also reasonable to assume that the per capita growth is a function of $P$ itself
\begin{displaymath}
\frac{dP}{dt} = kP\left(1-\frac{P}{M} \right) 
\end{displaymath}

where $k>0$ and $M>0$ are constants is called the {\bf logistic equation}.

\vspace{0.1in}
	

\pagebreak

Solution of initial value problem	
	
\begin{displaymath}
  \left\{ \begin{array}{ll}
  
  \frac{dP}{dt} = kP\left(1-\frac{P}{M} \right) \\
P(0) = P_0 \\
\end{array} \right.
\end{displaymath}
	

\vspace{1in.}
\pagebreak

EXAMPLES:\\

Suppose a population follows the logistic growth with a carrying capacity of $M=5000$ and $k=0.05/$year.
\begin{enumerate}
	\item {If the population starts at 200, what is the population 3 years later?}
	\item{How long does it take for the population to reach size 4000?}
	\item{Compare these numbers to a population that grows at a \emph{constant} $0.05 /$year and starts at the same size.}
\end{enumerate}

\vspace{5in}
Another population experiences logistic growth with $M=9500$.  If the initial population is 1100 and five years later the population as grown to 2300, find the size of the population ten years from the beginning.


\pagebreak

Let $a(t)$, and $b(t)$ be two populations of bacteria that experience constant relative growth at rates $\alpha$ and $\beta$ (so $a$ and $b$ grow exponentially.)  Suppose that both populations live in the same evironment and that samples of the populations can only detect the \emph{proportion} of population $a$.  Let $p(t)$ be the proportion.

\begin{displaymath}
p(t)= \frac{a(t)}{a(t) + b(t)}
\end{displaymath}
\begin{enumerate}
	\item Show that $p$ follows the logistic equation.
	\item Find the solution of the logistic equation by using the formulas for $a$  and $b$.
	
\end{enumerate}



\pagebreak

An contagious disease is spreading among a population.  Let $S(t)$ be the number of individuals who are susceptible to the disease, and $I(t)$ be the number who are infected.  One possible model for the spread of the disease is known as the $SI$-model.

\begin{eqnarray*}
\frac{dS}{dt} & = & -kSI \\
\frac{dI}{dt} & = & kSI
\end{eqnarray*}
The per capital change of each population is proportional to the other.  Determine a formula for $I(t)$ by showing that it follows logistic growth.



\pagebreak

A model for a population that experiences a constant removal rate $H$ can be found by making a modification to the logistic equation.  (\emph{The removal
	may represent fish or plants being harvested from the population.}) 

\begin{displaymath}
\frac{dP}{dt} = kP\left(1-\frac{P}{M} \right) - H
\end{displaymath}


\vspace{5in}
If $M=20000$, $k=0.03/$year, and $H = 150$ individuals/year, what is the lowest population that could survive such harvesting?
%\pagebreak
%\begin{center}
%\Large
%\rm{Chapter 9.3:  Separable Differential Equations}\\
%
%\end{center}
%
%A differential equation is called {\bf separable} if can be written in the form
%
%\begin{displaymath}
%  \frac{dy}{dx} = g(x)f(y)
%\end{displaymath}
%
%\vspace{0.2in}
%Solutions can be found by ``separating'' the variables and then integrating.\\
%
%\vspace{2.5in}
%
%EXAMPLES:
%
%\begin{displaymath}
%  \frac{dy}{dx} = ky
%\end{displaymath}
%
%\vspace{2in}
%
%\begin{displaymath}
%  \frac{dy}{dx} = 3x^2y
%\end{displaymath}
%
%\pagebreak
%
%Solve the initial value problems:\\
%
%\begin{displaymath}
%  \left\{ \begin{array}{ll}
%  \frac{dy}{dt} = y^2\sin{x} \\
%y(0) = 4 \\
%\end{array} \right.
%\end{displaymath}
%
%\vspace{2in}
%
%\begin{displaymath}
%  \left\{ \begin{array}{ll}
%  \frac{dy}{dt} = 3-4y\\
%y(0) = 10 \\
%\end{array} \right.
%\end{displaymath}
%
%\vspace{2.5in}
%
%
%\begin{displaymath}
%  \left\{ \begin{array}{ll}
%  \frac{dy}{dx} = \frac{-x}{y} \\
%y(1) = -3 \\
%\end{array} \right.
%\end{displaymath}
%
%
%\vspace{2in}
%
%NOTE:  It may not always be possible to ``solve'' explicity for $y(x)$.
%
%\begin{displaymath}
%  \frac{dy}{dx} = \frac{x^2}{1-y^2}
%\end{displaymath}
%
%
%\vspace{3.5in}
%
%APPLICATIONS:\\
%
%Solve the initial value problem for the velocity of a falling object that is subject to a drag force
%
%\begin{displaymath}
%  \left\{ \begin{array}{ll}
%  m\frac{dv}{dt} = g - kv\\
%v(0) = v_0 \\
%\end{array} \right.
%\end{displaymath}
%
%\pagebreak
%
%
%A tank contains 200 L of water with 0.8 kg of salt.   At $t=0$, a solution with concentration 1 g/L begins
%flowing into the tank at a rate of 2 L/min. If the tank is kept well-mixed and is drained at a rate of 2 L/min,
%find the amount of salt in the tank as a function of time.
%
%
%\vspace{4.5in}
%
%Find the family of curves that is orthogonal to every member of the family $y=\frac{k}{x}$.
%
%\pagebreak
%
%Let $h(t)$ and $V(t)$ be the height and volume of water in a tank at time $t$.  If the tank drains through a hole
%at the bottom with radius $a$, then Torrecelli's Law is the following equation.
%
%\begin{displaymath}
%  \frac{dV}{dt} = -a\sqrt{2gh}
%\end{displaymath}
%
%\vspace{1in}
%

\end{document}




