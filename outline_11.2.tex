\documentclass[11pt]{article}
\usepackage[letterpaper, margin=1in]{geometry}
\usepackage{amsmath, amssymb, graphicx, epsfig, fleqn}
\setlength{\parindent}{0pt}
\newcommand{\ud}{\,\mathrm{d}}
\everymath{\displaystyle}
\def\FillInBlank{\rule{2.5in}{.01in} }
\pagestyle{empty}

\begin{document}
\begin{center}
\Large
\rm{Math 112}
\\
\rm{Chapter 11.2:  Series}
\\
\end{center}
\vspace{0.2in}
\fboxsep0.5cm


A {\bf series} is an infinite sum of numbers.\\

\begin{displaymath}
\sum_{n=1}^{\infty} a_n
  \end{displaymath}

\vspace{.3in}



EXAMPLES:\\


$\sum_{n=1}^{\infty} \frac{1}{n}$

\vspace{0.3in}

$\sum_{n=2}^{\infty}  \frac{\ln{n}}{n^2}$

\vspace{0.3in}

$\sum_{n=0}^{\infty} \left(\frac12\right)^n $

\vspace{0.3in}


To understand if the series are finite, we define the sequence of {\bf partial sums} 

\begin{displaymath}
s_n = \sum_{k=0}^{n} a_k
  \end{displaymath}
\vspace{1.75in}

If the limit of the $s_n$ exists we say that the series $\sum a_n$ {\bf converges}. \\
If the limit of the $s_n$ does not exist we say that the series $\sum a_n$ {\bf diverges}.

\pagebreak

TELESCOPING SERIES\\

\begin{displaymath}
\sum_{n=1}^{\infty} \frac{1}{n(n+1)}
  \end{displaymath}



\vspace{2.5in}

GEOMETRIC SERIES

\begin{displaymath}
\sum_{n=0}^{\infty} r^n 
  \end{displaymath}

\pagebreak

EXAMPLES\\

$20 + 5 + \frac54 + \frac{5}{16}+ ...$

\vspace{2in}


$6 + 9 + \frac{27}{2} + \frac{81}{4}+ ...$


\vspace{2in}


$\sum_{n=1}^{\infty}\frac{2^{n-3}}{5^n}$
	
	\vspace{2in}

$1 - \frac17 + \frac{1}{7^2} - \frac{1}{7^3}+ ...$


\vspace{2in}

$\sum_{n=0}^{\infty}\frac{1}{\pi^n}$

\vspace{1.5in}

For what values of $x$ does the series converge?\\

$\sum_{n=1}^{\infty}\frac{(x-3)^n}{5^n}$



\vspace{2in}

For what values of $x$ does the series converge?\\

$\sum_{n=0}^{\infty}\left(\frac{2}{x}\right)^n$

\vspace{2in}

\pagebreak

REPEATING DECIMAL EXPANSIONS \\

\vspace{0.15in}


Use a geometric series to find the fraction represented by 0.72727272....\\

\vspace{2.5in}

Use a geometric series to find the fraction represented by 0.135135...\\

\vspace{3in}

Use a geometric series to find the fraction represented by 0.999...\\


\pagebreak

GENERAL SERIES\\

For most series, it is difficult or impossible to find a formula for the partial sums.  We will instead
rely on a series of {\bf convergence tests}.  These tests are rules that we can apply to determine if a series converges. \\   

\vspace{.25in}

THEOREM: If the series $\sum_{n=1}^{\infty} a_n $ converges, then $\lim_{n\to\infty}a_n = 0$.\\

\vspace{1in}
 
 {\bf Test for Divergence:}  If $\lim_{n\to\infty}a_n \neq 0$, then $\sum_{n=1}^{\infty} a_n $ diverges.\\
 
 \vspace{1in}
 
 EXAMPLES:\\
 
 $\sum_{n=1}^{\infty} \sqrt{n} $
 
 \vspace{1in}
 
  $\sum_{n=1}^{\infty} \frac{n^2}{2n^2+5} $
 
 \vspace{1in} 
 
 $\sum_{n=1}^{\infty} \frac{n}{\ln{n}} $
 
 \vspace{1in} 
 
 $\sum_{n=1}^{\infty} e^{1/n} $
 
 \vspace{1in}
 
 WARNING:  If $\lim_{n\to\infty}a_n = 0$, then $\sum_{n=1}^{\infty} a_n $ may converge \emph{or} diverge.\\
 
 \vspace{.15in}
 
 EXAMPLES:\\
 
 $\sum_{n=1}^{\infty} \frac{1}{2^n} $
 
  \vspace{1.5in}
 
$\sum_{n=1}^{\infty} \frac{1}{n} $

\vspace{1in}




\end{document}


