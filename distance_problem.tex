\documentclass[11pt]{article}
\usepackage[letterpaper, margin=1in]{geometry}
\usepackage{amsmath, amssymb, graphicx, psfrag, epsfig, fleqn}
\setlength{\parindent}{0pt}
\newcommand{\ud}{\,\mathrm{d}}
\everymath{\displaystyle}
\def\FillInBlank{\rule{2.5in}{.01in} }
\pagestyle{empty}

\begin{document}
\begin{center}
\Large
\rm{Math 112:  Distance Problem}
\\
\end{center}
\vspace{0.2in}
\fboxsep0.5cm

We are provided with velocity data for a moving object over the time interval $0\leq t\leq 20$.  \\ We would like to estimate the distance the object has traveled in this time interval.

\vspace{0.5cm}
\begin{tabular}{|c| c| c| c| c| c| c|} \hline
$t$ (s) & 0 & 5 & 10 & 15 & 20  \\ 
\hline
$v(t)$ (m/s) & 25 & 32 & 36 & 30 & 22  \\ 
\hline
\end{tabular}

\vspace{0.5cm}

\begin{enumerate}
\item {Estimate the distance that the object traveled in the first 5 seconds.}
\item {What assumption(s) did you make to produce your estimate?}
\item {Estimate the distance that the object traveled over the whole 20 second interval.}
\item {What assumption(s) did you make to produce your estimate?}
\item {What information is needed to determine the distance traveled more accurately?}
\item {Draw a picture to illustrate the connection between the distance problem and the area problem.}
\end{enumerate}

\end{document}
