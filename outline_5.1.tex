\documentclass[11pt]{article}
\usepackage[letterpaper, margin=1in]{geometry}
\usepackage{amsmath, amssymb, graphicx, epsfig, fleqn}
\setlength{\parindent}{0pt}
\newcommand{\ud}{\,\mathrm{d}}
\everymath{\displaystyle}
\def\FillInBlank{\rule{2.5in}{.01in} }
\pagestyle{empty}

\begin{document}
\begin{center}
\Large
\rm{Math 112}
\\
\rm{Chapter 5.1:  The Area Problem}
\\
\end{center}
\vspace{0.2in}
\fboxsep0.5cm

Our goal is to determine a way to calculate areas that are enclosed by curved boundaries.  Our starting point is an area
defined by $a \leq x \leq b$, and $0 \leq y \leq f(x)$  where $f$ is a positive function.\\

\vspace{2.2in}


How should we define the area?  What do we really \emph{know} about area? \\

\vspace{1.in}


(EXAMPLE) \\

Determine the area defined by $ 0 \leq x \leq 1$  and $0 \leq y \leq 1-x^2$

\vspace{2.5in}


       {\bf Idea:} Divide the area into pieces and estimate the area of each piece with a simple shape.


       
\pagebreak
{\bf $L_2$:} Area of 2 rectangles with height determined by the function value on the left side.       
       
\vspace{2.5in}

{\bf $L_4$:} Area of 4 rectangles with height determined by the function value on the left side.       

\vspace{2.5in}


{\bf $L_8$:} Area of 8 rectangles with height determined by the function value on the left side.       

\vspace{1.5in}

What happens as the number of rectangles gets larger?


\pagebreak

Another possibility:\\

{\bf $R_2$:} Area of 2 rectangles with height determined by the function value on the left side.       
       
\vspace{2.5in}

{\bf $R_4$:} Area of 4 rectangles with height determined by the function value on the left side.       

\vspace{1.5in}

What happens as the number of rectangles gets larger?

\vspace{1.5in}

{\bf Idea:} It may be possible to define the area as a limit on the area as the number of rectangles grows!

\vspace{.5in}
Formula for $L_N$:

\pagebreak


Handy formula:  $1^2 + 2^2 + 3^2 + ... + M^2 = \frac{M(M+1)(2M+1)}{6}$

\vspace{3.5in}

$ A = \lim_{N\to\infty}L_N$ 

\vspace{2.5in}


(EXERCISES)\\
\begin{enumerate}
\item{Find a formula for $R_N$ and calculate $\lim_{N\to\infty}R_N$.  You will get the same number}
\item{Repeat the general procedure to find the area bounded defined by $0\leq x \leq 2$ and $0 \leq y \leq x^3$\\
You will need a handy formula $1^3 + 2^3 + 3^3 + ... + M^3 = \frac{M^2(M+1)^2}{4}$.}
\end{enumerate}

\pagebreak


\begin{center}
\Large
\rm{Sigma (Summation) Notation}
\\
\end{center}
Sigma notation is a compact way to write sums of numbers when there is a pattern in the numbers.
\begin{displaymath}
\sum_{i=m}^na_i = a_m + a_{m+1} + ... + a_{n-1} + a_n
  \end{displaymath}
\vspace{0.2in}

(EXAMPLES)\\
$\sum_{i=1}^4 i^2$ \\

  \vspace{0.2in}
  $\sum_{i=3}^7 i$\\
  
  \vspace{0.2in}
  $\sum_{i=0}^5 2^1$\\
  
  \vspace{0.2in}
$\sum_{i=2}^6 3 $ \\

    \vspace{0.2in}

    (NOTES)\\

    \vspace{3in}


    Writing $L_N$ and $R_N$ from the previous example.


\end{document}


