\documentclass[11pt]{article}
\usepackage[letterpaper, margin=1in]{geometry}
\usepackage{amsmath, amssymb, graphicx, epsfig, fleqn}
\setlength{\parindent}{0pt}
\newcommand{\ud}{\,\mathrm{d}}
\everymath{\displaystyle}
\def\FillInBlank{\rule{2.5in}{.01in} }
\pagestyle{empty}

\begin{document}
\begin{center}
\Large
\rm{Math 112}
\\
\rm{Practice Problems}
\\
\vspace{0.2in}

\end{center}


  \begin{enumerate}

  \item{Write down the $R_N$ approximation for the area below the graph of $f(x)=x^2+x$ and above the $x$-axis for $1\leq x \leq 4$.  Use summation notation to write $R_N$.}

  \item{Write $R_N$ from the previous problem in terms of $N$.\\
    (\emph{Hint: You will need the handy formulas from the notes.})}
    
%\item{If $y(2)=15$, $y'$ is continuous, and $\int_2^5 y'(x)\,dx = 10$, what is the value of $y(5)$}
    
\item{If  $\int_2^5 y(x)\,dx = 10$, what is the value of $\int_2^5 3y(x)-6\,dx$}

\item{If $g(x) = \int_{x}^{\sqrt{x}} 2-e^{-t}\,dt$, find $g'(9)$.}

    
  \item{Evaluate the integrals:

    $\int \frac{x}{e^{x}} \, dx$

\vspace{0.2in}
    
        $\int_0^{3} |x^2+2x-3| \, dx$

\vspace{0.2in}
    
        $\int_0^{1/2} \theta\tan{(\pi\theta^2)} \, d\theta$

\vspace{0.2in}
    
        $\int \frac{xe^{2x}}{(1+2x)^2} \, dx$

\vspace{0.2in}
    
        $\int \frac{\sec^2{\theta}}{\tan^2{\theta}} \, dx$

\vspace{0.2in}
    
        $\int_{\pi/2}^{\pi} \sin^5{(3x)}\cos{(3x)} \, dx$

  }

    \vspace{0.1in}
    
\item{Find the average value of the function $h(x)=\ln{\sqrt{x}}$ on the interval $[1,16]$.}
    
\item{Calculate $\lim_{N\to\infty}(1/N) \sum_{i=1}^N\sqrt{1-(i/N)^2}$}

    
    \item{Let $A = \int_1^4 \ln{x}\, dx$.  Determine how large $N$ should be so that $R_N-A < 0.0001$ without using the value of $A$. \\



  (\emph{Hint: $L_N<A<R_N$ (why?) so $R_N-A < R_N-L_N$.  })
}


  \item{Determine the area of the region described by the inequalities $x-2y^2\ge 0$ and $1-x-|y|\ge 0$.  Determine the volume produced by rotating the area around the line $x=1$.}
  \item{Find the number $b$ so that the line $y=b$ divides the region bounded by the curves $y=x^2$ and $y=4$ into
    two equal areas.}
  \item{For what values of $m$ do the line $y=mx$ and the curve $y=\frac{x}{x^2+1}$ enclose a region?  Find the area
    of that region in terms of $m$.}
    

  \item{Describe the volume represented by the integral $\int_1^3 2\pi y\ln{y} \, dy$.}

  \item{Suppose $g$ is a function that is increasing and concave up on $[a,b]$. \\ Which is greater, $\bar{g}$ or $g\left(\frac{a+b}{2}\right)$?  Why? 
  \emph{Hint: Draw a picture.}  }

      \end{enumerate}



\end{document}


