\documentclass[11pt]{article}
\usepackage[letterpaper, margin=1in]{geometry}
\usepackage{amsmath, amssymb, graphicx, epsfig, fleqn}
\setlength{\parindent}{0pt}
\newcommand{\ud}{\,\mathrm{d}}
\everymath{\displaystyle}
\def\FillInBlank{\rule{2.5in}{.01in} }
\pagestyle{empty}

\begin{document}
\begin{center}
\Large
\rm{Math 112}
\\
\rm{Chapter 6.5:  Integrals and Averages}
\\
\end{center}
\vspace{0.2in}
\fboxsep0.5cm

If we have a set of data $x_1$, $x_2$, $x_3$,  ...  $x_N$, we define the average $\bar{x}$ as:\\

\vspace{0.25in}


EXAMPLE:\\

In order to calculate the average outside temperature over a period of 10 hours, we could collect data at a set of times.\\


\vspace{1.5in}

Calcuate $\bar{T}$:\\

\vspace{.25in}

For a more accurate average, we can collect more data.

\vspace{2.5in}

If we let the number of data points $N$ go to infinity, we can produce an integral

\pagebreak

DEFINITION:\\

The {\bf average value} of a function $f$ on an interval $[a,b]$ is $\bar{f}$\\

\vspace{1.5in}


EXAMPLE: \\

Find the average value of $f(x) = \sqrt[3]{x} $ on the interval $[0,8]$.\\

\vspace{1.5in}

APPLICATION:\\

A growing fish has length that is modeled by the function $L(t) = 20 - \frac{20}{t+1}$.\\
\begin{enumerate}
\item{What is the average length of the fish over the first 10 years of its life?}
\item{What is the average length of the fish over the first $T$ years of its life?}
\item{What happens to this average as $T$ gets large?}

\end{enumerate}


\vspace{2.8in}

APPLICATION:\\

The velocity of fluid flowing slowly in a cylindrical tube of radius $R$ is a function of the radial distance from the center.

\begin{displaymath}
v(r) = \frac{P}{4\eta L}(R^2 - r^2)
  \end{displaymath}

Find the average velocity from $r=0$ in the center to $r=R$ on the edge.  Make a comparision to the maximum velocity.\\

\vspace{3.5in}


Geometric meaning:


\pagebreak

EXAMPLE:\\

Find a value of $x^*$ in $[0,\pi]$ so that $\int_0^{\pi}\sin{x} \, dx = \pi\sin{x^*}$.

\vspace{2.5in}

{\bf Mean Value Theorem for Integrals} \\

If $f$ is a continuous function on $[a,b]$, then there exists a number $c$ in $[a,b]$ such that \\
\begin{displaymath}
\int_a^b f(x) \, dx = f(c)(b-a)
\end{displaymath}  

\vspace{1.7in}

  EXAMPLE:\\

  Find the value of $c$ described in the theorem for $f(x) = e^x $ on the interval $[-1,3]$.



\end{document}


