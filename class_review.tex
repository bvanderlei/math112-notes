\documentclass[11pt]{article}
\usepackage[letterpaper, margin=1in]{geometry}
\usepackage{amsmath, amssymb, graphicx, psfrag, epsfig, fleqn}
\setlength{\parindent}{0pt}
\newcommand{\ud}{\,\mathrm{d}}
\everymath{\displaystyle}
\def\FillInBlank{\rule{2.5in}{.01in} }
\pagestyle{empty}

\begin{document}
\begin{center}
\LARGE
\rm{Math 112:  Review}
\\
\end{center}
\vspace{0.2in}
\fboxsep0.5cm
{\bf Integrals}

\begin{enumerate}
\item {Give the definition of the {\bf definite integral} and explain its meaning.  Give an example of a definite integral and sketch a picture
to interpret what it means geometrically.}
\item {Explain the meaning of an {\bf indefinite integral}.  Give an example.}
\item {Explain how the {\bf Fundamental Theorem of Calculus} connects definite and indefinite integrals.  Give an example.}
\item {Give three examples of {\bf applications for integrals}.  Provide as many details as possible.  Give specific integrals, functions, units, etc.}
\item {Provide examples of three {\bf methods for integration}.  Give specific examples if you can.}
\item {Explain the need for {\bf numerical integration}.  Give one such example and draw a picture to demonstrate how the method works.}
\item {Give examples of the two types of {\bf improper integrals}.  Explain what it means for an improper integral to converge.  Describe how to determine if an improper integral converges.}
\end{enumerate}


{\bf Series}

\begin{enumerate}
\item {Explain the difference between a {\bf sequence} and a {\bf series}.}
\item {Explain what it means for a {\bf sequence} to converge.}
\item {Explain what it means for a {\bf series} to converge.}
\item {Give an example of a {\bf geometric series} that converges and find the value of the series.}
\item {Give an example of a {\bf  $p$-series }that converges and describe how we might determine that it converges.}
\item {Give three examples of {\bf convergence/divergence tests}, and how they are used on a specific series.}
\item {Explain what is meant by a {\bf power series}.  Give two reasons we may want to represent a function as a power series.}
\item {Show how to compute the {\bf Taylor series} for a specific function.}
\end{enumerate}


{\bf Differential Equations}
\begin{enumerate}
\item {Give an example of a {\bf differential equation} and the {\bf direction field} it generates.}
\item {Explain how to approximate solutions to differential equations using  {\bf Euler's Method}.  Give an example.}
\item {Give an example of a {\bf separable differential equation} and show how to find the solution.}
\item {Describe solutions to the  {\bf logistic equation}.  Give two reasons why the logistic equation might be a good model of a changing population.}
\item {Compare solutions to the logistic equation with solutions to the law of natural growth.  Explain what is meant by {\bf relative growth rate} in these models.}

\end{enumerate}

\end{document}
