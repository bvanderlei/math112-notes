\documentclass[11pt]{article}
\usepackage[letterpaper, margin=1in]{geometry}
\usepackage{amsmath, amssymb, graphicx, epsfig, fleqn}
\setlength{\parindent}{0pt}
\newcommand{\ud}{\,\mathrm{d}}
\everymath{\displaystyle}
\def\FillInBlank{\rule{2.5in}{.01in} }
\pagestyle{empty}

\begin{document}
\begin{center}
\Large
\rm{Math 112}
\\
\rm{Chapter 11.1:  Sequences}
\\
\end{center}
\vspace{0.2in}
\fboxsep0.5cm


A {\bf sequence} as a list of numbers in a definite order.  Our goal is to understand infinite sequences and their limits.

\vspace{0.7in}

NOTATION:

\vspace{0.7in}


EXAMPLES:\\


$\left\{\frac12, \frac23, \frac34, \frac45, ... \right\}$

\vspace{0.7in}

$\left\{\frac{(-1)^{n+1}}{n} \right\}_{n=1}^{\infty}$

\vspace{0.7in}

$\left\{\cos{\left(\frac{n\pi}{4}\right)} \right\}_{n=0}^{\infty}$

\vspace{0.7in}

EXERCISES:

Find a formula for $a_n$ in terms of $n$\\

$\left\{3, \frac32, \frac34, \frac38,  ... \right\}$

\vspace{0.7in}

$\left\{\frac{-1}{6}, \frac{3}{10}, \frac{-5}{14}, \frac{7}{18}, ... \right\}$

\pagebreak

If we can write a formula for $a_n$ in terms of $n$, we can think of the sequence as a function with domain $\left\{1, 2, 3, 4,  ... \right\}$.

\vspace{2in}


Some sequences may be easier to describe with recurrence relations (later terms are related to earlier terms)\\

EXAMPLES:\\

$a_1= 0, \quad a_2=1, \quad a_n=\frac{-a_{n-2}}{n^2}$

\vspace{2in}


$a_1= 1, \quad a_2=1, \quad a_n= a_{n-1}+a_{n-2} \quad\quad$  (Fibonacci sequence)


\vspace{2in}

Let $C_n$ be the amount of caffeine in a person's blood stream in units of bce (Ben's coffee equivalent).  It is known that after
one hour the body will have absorbed 13\% of the caffiene.  Suppose that Ben starts his morning with one coffee ($C_0 = 1$) and at the end of each hour quickly drinks another coffee.

\pagebreak

We define the {\bf limit} of a sequence similar to the way we define the limit of a function at infinity.\\

\begin{displaymath}
  \lim_{n\to \infty}a_n = L
\end{displaymath}

means that all $a_n$ for which $n > N$ are arbitrarily close to $L$ when $N$ is sufficiently large.  We say  that $L$ is the {\bf limit} of the sequence.\\

If the limit exists we say that the sequence {\bf converges}. \\
If the limit does not exist we say that the sequence {\bf diverges}.


\vspace{.5in}

(EXAMPLES)\\

$a_n=\frac{3}{2^{n-1}}$

\vspace{.7in}


$b_n=\frac{2n}{3n^2+1}$

\vspace{.7in}

$c_n=\cos{\left(\frac{n\pi}{4}\right)}$

\vspace{.7in}

$d_n=n^2$

\vspace{.7in}

$a_n=\frac{\ln{n}}{n}$




\pagebreak

THEOREM:  If $\lim_{n\to\infty}|a_n| = 0$, then $\lim_{n\to\infty}a_n = 0$\\

\vspace{.1in}

EXAMPLE:\\

$a_n=\frac{(-1)^{n+1}n^2}{2n^4+5}$

\vspace{1in}

$b_n=\frac{(-1)^{n+1}n^2}{2n^2+5}$


\vspace{1.2in}

For what values of $r$ does $r^n$ converge?

\vspace{1.5in}

OTHER SEQUENCES:\\

$a_n=\frac{n!}{2^n}$

\vspace{1.2in}




$\left\{1, 0, 1, 0, 0, 1, 0, 0, 0, 1, 0, 0, 0, 0, 1,   ... \right\}$

\pagebreak

But what happens to Ben!?  $C_n = 0.87C_{n-1} + 1$

\vspace{5in}

MONOTONE CONVERGENCE THEOREM:  Every bounded monotone sequence is convergent.

\pagebreak



DYNAMICAL SYSTEMS: \\

Let $P_t$ represent the population of species in a given habitat in year $t$.  One simple way to model the population is to suppose that the population in the future year, $P_{t+1}$ is a function of $P_{t}$, the population during the current year.\\

\begin{displaymath}
P_{t+1} = 1.07P_t
  \end{displaymath}

\vspace{2.5in}

\begin{displaymath}
P_{t+1} = \frac{1000}{200+P_t}P_t
  \end{displaymath}




\end{document}


