\documentclass[11pt]{article}
\usepackage[letterpaper, margin=1in]{geometry}
\usepackage{amsmath, amssymb, graphicx, epsfig, fleqn}
\setlength{\parindent}{0pt}
\newcommand{\ud}{\,\mathrm{d}}
\everymath{\displaystyle}
\def\FillInBlank{\rule{2.5in}{.01in} }
\pagestyle{empty}

\begin{document}
\begin{center}
\Large
\rm{Math 112}
\\
\rm{Chapter 6.2:  Volumes}
\\
\end{center}
\vspace{0.2in}
\fboxsep0.5cm

The goal of this section is to use integrals to compute volumes.  We
start by looking at \\ {\bf volumes of revolution}.\\

\vspace{3in}

$V = \int_a^b A(x) \, dx$   where $A(x)$ is the area of the cross-section at $x$.

\vspace{0.5in}


EXAMPLES: 

\begin{enumerate}
  \item{An area is bounded by $y=0$, $y=\sqrt{x}$, and $x=1$.  Find the volume generated by rotating the area around the $x$-axis.}


  \pagebreak

\item{An area is bounded by $y=0$, $y=\sec{x}$, $x=-\pi/6$ and $x=\pi/6$.  Find the volume generated by rotating the area around the $x$-axis.}

  \vspace{4in}


\item{An area is bounded by $y=x^3$, $y=8$, and $x=0$.  Find the volume generated by rotating the area around the $y$-axis.}  


\pagebreak

\item{Find the volume of solid obtained by rotating the the area bounded by $y=x^2+1$ and  $y=9-x^2$ around the $x$-axis.}  


  \vspace{4in}


\item{Find the volume of solid obtained by rotating the the area bounded by $x=y^2+1$ and  $y=x-3$ around the $y$-axis.}

  \pagebreak

\item{The volume of a sphere of radius $r$ is $V=\frac43 \pi r^3$.  Derive this formula with an integral.}

  \vspace{4in}


\item{The volume of a cone with a height $h$ and base radius $r$ is $V=\frac13 \pi r^2h$.  Derive this formula with an integral.}  

  \pagebreak

  
\end{enumerate}



\end{document}


