\documentclass[11pt]{article}
\usepackage[letterpaper, margin=1in]{geometry}
\usepackage{amsmath, amssymb, graphicx, epsfig, fleqn}
\setlength{\parindent}{0pt}
\newcommand{\ud}{\,\mathrm{d}}
\everymath{\displaystyle}
\def\FillInBlank{\rule{2.5in}{.01in} }
\pagestyle{empty}

\begin{document}
\begin{center}
\Large
\rm{Math 112}
\\
\rm{Chapter 9.1:  Differential Equations}\\

\end{center}
\vspace{0.2in}
\fboxsep0.5cm

\vspace{0.2in}

A {\bf differential equation} is a relationship between and unknown function and its derivatives.

\vspace{0.2in}

EXAMPLES:

\begin{displaymath}
  \frac{dy}{dt} = 4y
\end{displaymath}
\vspace{0.8in}

\begin{displaymath}
  \frac{dy}{dt} = 3t^2y
\end{displaymath}
\vspace{0.8in}

\begin{displaymath}
  \frac{dy}{dt} = \cos{t}
\end{displaymath}
\vspace{0.8in}
\begin{displaymath}
  \frac{dy}{dt} + y = e^{-t}
\end{displaymath}
\vspace{0.8in}
NOTES:

\pagebreak

If we attach an {\bf initial condition} to the differential equation, we get what is called an {\bf initial value problem}.\\

\begin{displaymath}
  \left\{ \begin{array}{ll}
  y' = \cos{t} \\
y(0) = 6 \\
\end{array} \right.
\end{displaymath}

\vspace{0.2in}

A differential equation alone has a family of solutions, a well-posed initial value problem has only one solution.\\

\begin{displaymath}
  \left\{ \begin{array}{ll}
  y' + y = e^{-t} \\
y(0) = 2 \\
\end{array} \right.
\end{displaymath}

\vspace{2in}

EXAMPLES OF DIFFERENTIAL EQUATIONS AS MODELS:\\

\begin{enumerate}
\item{Let $P(t)$ represent the size of a growing population.  Determine an equation for $P$ if it is assumed that rate
  at which the population grows is proportional to its size.}

\pagebreak

\item{An object falls from a given height with only the force of gravity acting upon it.  Find a differential equation
for $v(t)$, its velocity as a function of time.}

  \vspace{2in}

\item{An object falls from a given height with only the force of gravity acting upon it.  Find a differential equation
for $h(t)$, its \emph{height} as a function of time.}

  \vspace{2.5in}

\item{An object falls from a given height, and in addition to the force of gravity, it experiences a drag force
that is proportional to its velocity.  Find a differential equation for the velocity $v(t)$.}

\pagebreak
  
  \vspace{2in}
\item{A mass is attached to one end of a spring, while the other end is attached to a wall.  The force needed to stretch
  the spring is proportional to the length it is stretched from equilibrium.  Let $x(t)$ be the position of the mass and
find a differential equation for $x(t)$.}
  
  

\end{enumerate}


\pagebreak

\begin{center}
\Large
\rm{Direction Fields}\\

\end{center}
\vspace{0.2in}

If a differential equation can be written in the form

\begin{displaymath}
  \frac{dy}{dx} = f(x,y)
\end{displaymath}

\vspace{0.1in}

we say that $f$ is the rate function.  We can use $f$ to draw a {\bf direction field}.\\

\vspace{0.3in}


EXAMPLES:


\begin{displaymath}
  \frac{dy}{dx} =  y + x
\end{displaymath}

\pagebreak



\begin{displaymath}
  \frac{dy}{dx} =  4-2y
\end{displaymath}
\vspace{1.in}

\vspace{3in}

\begin{displaymath}
  \frac{dy}{dx} =  y\cos{x}
\end{displaymath}
\vspace{1.in}


\end{document}




