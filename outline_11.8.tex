\documentclass[11pt]{article}
\usepackage[letterpaper, margin=1in]{geometry}
\usepackage{amsmath, amssymb, graphicx, epsfig, fleqn}
\setlength{\parindent}{0pt}
\newcommand{\ud}{\,\mathrm{d}}
\everymath{\displaystyle}
\def\FillInBlank{\rule{2.5in}{.01in} }
\pagestyle{empty}

\begin{document}
\begin{center}
\Large
\rm{Math 112}
\\
\rm{Chapter 11.8:  Power Series}
\\
\end{center}
\vspace{0.2in}
\fboxsep0.5cm

A {\bf power series} is a series that has the form

  \begin{displaymath}
\sum_{n=0}^{\infty}c_n(x-a)^n
  \end{displaymath}
  where $x$ is a variable, $a$ is a constant, and the sequence of values $c_n$ are constants.  The series is considered a function of $x$.
  \vspace{0.3in}

 EXAMPLE:\\

 $\sum_{n=1}^{\infty} \frac{(x-3)^n}{n}$

 \vspace{3in}

 THEOREM:  For a power series  $\sum_{n=0}^{\infty}c_n(x-a)^n$, there are three possibilities:

 \begin{enumerate}
 \item{The series converges only when $x=a$.}
 \item{The series converges for all $x$.}
 \item{There is a positive number $R$ such that the series converges if $|x-a|<R$ and diverges if $|x-a|>R$.}
 \end{enumerate}

 $R$ is called the {\bf radius of convergence}.
 

 \pagebreak

 EXAMPLES:  (For each series, find the radius of convergence and the {\bf interval of convergence}.)\\

 

 
 $\sum_{n=1}^{\infty} \frac{4^nx^n}{n^2}$

 \vspace{2.8in}

 $\sum_{n=0}^{\infty} \frac{(-1)^{n+1}x^{2n}}{2^{2n}(n!)^2}$


 \vspace{3.1in}

 $\sum_{n=0}^{\infty} x^n(n!)$
 
 \pagebreak

 $\sum_{n=2}^{\infty} \frac{x^{2n}}{n(\ln{n})^3}$
 
 
 \vspace{3in}


 $\sum_{n=1}^{\infty} \frac{(2x-1)^n}{5^n\sqrt{n}}$

 \vspace{3in}

 $\sum_{n=0}^{\infty} \frac{n^2}{6^{n+2}}(x+3)^n $

\pagebreak

EXERCISES:\\

For each series, find the radius of convergence and the interval of convergence.\\


 $\sum_{n=0}^{\infty} \frac{(x-2)^n}{n^2+4} $

  \vspace{1.5in}
  
    $\sum_{n=1}^{\infty} n^nx^n$  

  \vspace{1.5in}
  
    $\sum_{n=2}^{\infty} \frac{(x-5)^n}{3^n\ln{n}}$  


    \vspace{1.5in}
  
    $\sum_{n=0}^{\infty} \frac{(n!)^k}{(kn)!}x^n$ \quad ($k$ is some positive integer)  


    \vspace{1in}
    Suppose $\lim_{n\to\infty}\sqrt[n]{|c_n|}=C$.  Explain why the radius of convergence for $\sum c_nx^n$ is $\frac1C$.

    \pagebreak
    
\begin{center}
\Large
\rm{Chapter 11.9:  Power Series Construction}
\\
\end{center}
\vspace{0.5in}

The goal of this section is to find ways to represent some functions as power series be relating them
to the sum of the geometric series.
\begin{displaymath}
\sum_{n=0}^{\infty}x^n = 1 + x + x^2 + x^3 + .... = \frac{1}{1-x} \quad\quad \mbox{for} \,|x|< 1
  \end{displaymath}

\vspace{.2in}

EXAMPLES:\\


In each case, the goal is to write the function as a power series of the form $\sum c_nx^n$.\\


$\frac{1}{1+x}$

\vspace{1.5in}

$\frac{x^3}{1+x^2}$

\vspace{1.5in}

$\frac{1}{x+6}$


\pagebreak

THEOREM: \\
If the power series $\sum c_n(x-a)^n$ has radius of convergence $R>0$,
then the function $f$ defined by 

\begin{displaymath}
f(x) = c_0 + c_1(x-1) + c_2(x-a)^2 + .... = \sum_{n=0}^{\infty}c_n(x-a)^n 
  \end{displaymath}
is differentiable for $|x-a|<R$ and 
\begin{eqnarray*}
f'(x) &=& c_1 + 2c_2(x-a) +  3c_3(x-a)^2.... = \sum_{n=1}^{\infty}c_nn(x-a)^{n-1} \\
\int f(x)\,dx &=& C + c_0(x-a) + c_1\frac{(x-a)^2}{2} + c_2\frac{(x-a)^3}{3}...  = C + \sum_{n=0}^{\infty}c_n\frac{(x-a)^{n+1}}{n+1}
  \end{eqnarray*}
The radius of convergence for both of these new series is also $R$.

\vspace{.5in}

EXAMPLES: \\

In each case, the goal is to write the function as a power series of the form $\sum c_nx^n$.\\

$\frac{1}{(1-x)^2}$

\vspace{2in}

$\ln{|1+x|}$

\vspace{2in}

$\arctan{x}$


\end{document}




