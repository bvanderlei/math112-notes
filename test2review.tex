\documentclass[11pt]{article}
\usepackage[letterpaper, margin=1in]{geometry}
\usepackage{amsmath, amssymb, graphicx, epsfig, fleqn}
\setlength{\parindent}{0pt}
\newcommand{\ud}{\,\mathrm{d}}
\everymath{\displaystyle}
\def\FillInBlank{\rule{2.5in}{.01in} }
\pagestyle{empty}

\begin{document}
\begin{center}
\Large
\rm{Math 112}
\\
\rm{Practice Problems}
\\
\vspace{0.2in}

\end{center}


  \begin{enumerate}

  \item{Evaluate the integrals:

    \vspace{0.1in}
    
    $\int \frac{x+2}{x^2+3x-4} \, dx$

\vspace{0.2in}
    
        $\int \ln{(1+x^2)} \, dx$

\vspace{0.2in}
    
        $\int_0^{\pi} t\cos^2{t} \, dt$

\vspace{0.2in}
    
        $\int_0^4 \frac{z}{z-3} \, dz$  \hspace{0.8in} \mbox{Be careful.}

\vspace{0.2in}
    
        $\int_{0}^{\sqrt{2}/2} \frac{x^2}{\sqrt{1-x^2}} \, dx$

\vspace{0.2in}
    
     
$\int_0^2 x^3\sqrt{4x^2-x^4} \, dx$ \hspace{0.4in} \mbox{Use the formula} $\, \, \int \sqrt{a^2-u^2} \, du = \frac{u}{2}\sqrt{a^2-u^2}
+\frac{a^2}{2}\sin^{-1}{\frac{u}{a}} + C$
    
 
  }

    \vspace{0.1in}

\item{The table shows values of a force function $f(x)$.  Use Trapezoid, Midpoint, and Simpson's Rules to estimate the work done in moving an object from $x=0$ to $x=18.$}

  \vspace{0.15in}
  
\begin{tabular}{|c| c| c| c| c| c| c| c| c|} \hline
$t$ & 0 & 3 & 6 & 9 & 12 & 15 & 18  \\ 
\hline
$r(t)$ & 9.2 & 8.9 & 8.6 & 8.0 & 7.7 & 7.6 & 7.0 \\ 
\hline
\end{tabular}

\vspace{0.2in}

\item {Consider the following definite integral. 
\begin{displaymath}
\int_1^4 \frac{1}{\sqrt{x}}\, dx
\end{displaymath}
\begin{enumerate}
\item{Calculate the $T_6$, $M_6$, and $S_6$ approximations using Trapezoid, Midpoint, and Simpsons Rule.}
\item{How large should $N$ be so that the error in the Midpoint Rule approximation is less than 0.0005?}
\end{enumerate}
}

  \pagebreak
  
\item{If $f(t)$ is continuous for $t\ge 0$, the \emph{Laplace transform} of $f$ is defined as 
\begin{displaymath}
F(s) = \int_{0}^{\infty}f(t)e^{-st} \, dt.
\end{displaymath}

\begin{enumerate}
\item{Find a formula and domain for $F(s)$ if $f(t) = t$.}
  \item{Find a formula and domain for $F(s)$ if $f(t) = e^{2t}$.}
\end{enumerate}
}


  \item {Calculate the second Taylor polynomial, $T_2(x)$, for the function $f(x)=\sec{x}$ at $a=0$.  
  }

  \item{Find the sum of each series, or explain why the series diverges.\\

    $ 1 - \frac{2}{7} + \frac{4}{49} - \frac{8}{243} + ...$\\

        $ \frac{1}{5} + \frac{1}{9} + \frac{1}{13} +  \frac{1}{17} + ...$\\

    $ 1 + 2 + \frac{4}{2!} + \frac{8}{3!} + \frac{16}{4!} ...$\\

          $3 +  \frac{5}{3} + \frac{7}{5} + \frac{9}{7} ...$\\

    $\sum_{n=1}^{\infty}\ln{\left(1+\frac{1}{n}  \right)}$

    $\sum_{n=1}^{\infty}\frac{n}{(n+1)!}$
  }


  \item{     Explain why the radius of convergence for $\sum c_n(x-a)^n$ is $\lim_{n\to\infty}\left|\frac{c_n}{c_{n+1}} \right|$, if this
  limit exists.}


    \item{For what values of $x$ do the following series converge\\

      $\sum_{n=1}^{\infty}\frac{(-1)^n}{(2n-1)2^n}(x-1)^n$\\
      
      $\sum_{n=1}^{\infty}\frac{3^n}{x^n}$
    }

    \item{ Suppose $a$ and $b$ are real numbers with $a<b$.  Find a power series that converges on $(a,b]$.}
      
      \item{Find all positive values of $b$ for which the series $\sum_{n=1}^{\infty}b^{\ln{n}}$ converges.}
\end{enumerate}


\end{document}


