\documentclass[11pt]{article}
\usepackage[letterpaper, margin=1in]{geometry}
\usepackage{amsmath, amssymb, graphicx, epsfig, fleqn}
\setlength{\parindent}{0pt}
\newcommand{\ud}{\,\mathrm{d}}
\everymath{\displaystyle}
\def\FillInBlank{\rule{2.5in}{.01in} }
\pagestyle{empty}

\begin{document}
\begin{center}
\Large
\rm{Math 112}
\\
\rm{Chapter 11.5:  Alternating Series}
\\
\end{center}
\vspace{0.2in}
\fboxsep0.5cm

{\bf Alternating Series Test:}  \\

If the alternating series 
  \begin{displaymath}
    \sum_1^{\infty}(-1)^{n-1}b_n = b_1 - b_2 + b_3 - b_4 + ... \hspace{1in} (b_n\geq 0)
  \end{displaymath}
  satisfies $b_{n+1}\leq b_n $ for large $n$ and $\lim_{n\to \infty}b_n = 0$
  then the series converges.\\

  \vspace{0.2in}

 EXAMPLES:\\

$\sum_{n=1}^{\infty} \frac{(-1)^{n-1}}{n}$\quad (Alternating Harmonic Series)

\pagebreak

$\sum_{n=1}^{\infty} \frac{n^2(-1)^{n-1}}{n^3+1}$


\vspace{2.5in}

$\sum_{n=1}^{\infty} (-1)^{n+1}n^2e^{-4/n}$

\vspace{2.5in}

$\sum_{n=1}^{\infty} \frac{5n(-1)^{n-1}}{n^2+9}$

\pagebreak

{\bf Remainder Estimate for Alternating Series:}  \\

If the alternating series
  \begin{displaymath}
    \sum_1^{\infty}(-1)^{n-1}b_n = b_1 - b_2 + b_3 - b_4 + ... \hspace{1in} (b_n\geq 0)
  \end{displaymath}
converges to a number $s$, and $R_n= s-s_n$ is the remainder, then $|R_n|\leq b_{n+1}$.\\

EXAMPLES: \\

How large can $R_{10}$ be for the series $\sum_1^{\infty}\frac{(-1)^{n+1}}{n!}$?

\vspace{1.5in}

How large should $N$ be so that $\sum_1^{\infty}\frac{(-1)^{n+1}}{n^3}- \sum_1^{N}\frac{(-1)^{n+1}}{n^3} < 0.0001$?

\vspace{1.5in}

How large should $N$ be so that $\sum_2^{\infty}\frac{(-\frac12)^{n+1}}{n}- \sum_2^{N}\frac{(-\frac12)^{n+1}}{n} < 0.005$?

\pagebreak

DEFINITIONS:  \\

A series $\sum a_n$ is called {\bf absolutely convergent} if the series of absolute values $\sum|a_n|$ is convergent.\\

A series $\sum a_n$ is called {\bf conditionally convergent} if the series is convergent but not absolutely convergent.\\

EXAMPLES:\\

$\sum_{n=1}^{\infty} \frac{(-1)^{n-1}}{n}$\quad 

\vspace{1.5in}

$\sum_{n=1}^{\infty} \frac{(-1)^{n-1}}{n^2}$\quad 


\vspace{1.5in}

THEOREM:  If a series $\sum a_n$ is absolutely convergent, then it is convergent.\\

\vspace{.5in}

$\sum_{n=1}^{\infty} \frac{\cos{n}}{n^2}$\quad 

\pagebreak

EXERCISES:\\

$\sum_{n=1}^{\infty} \frac{(-1)^{n-1}\sqrt{n}}{3n+2}$

\vspace{1.8in}

$\sum_{n=1}^{\infty} (-1)^{n-1}\sin{\left( \frac{1}{n} \right)}$

\vspace{1.8in}

$\sum_{n=1}^{\infty} \frac{n\cos{(n\pi)}}{4^n}$

\vspace{1.8in}

$\frac{1}{\ln{3}} - \frac{1}{\ln{5}} + \frac{1}{\ln{7}} - \frac{1}{\ln{9}} + ....$




\end{document}




