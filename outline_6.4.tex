\documentclass[11pt]{article}
\usepackage[letterpaper, margin=1in]{geometry}
\usepackage{amsmath, amssymb, graphicx, epsfig, fleqn}
\setlength{\parindent}{0pt}
\newcommand{\ud}{\,\mathrm{d}}
\everymath{\displaystyle}
\def\FillInBlank{\rule{2.5in}{.01in} }
\pagestyle{empty}

\begin{document}
\begin{center}
\Large
\rm{Math 112}
\\
\rm{Chapter 6.2:  Volumes}
\\
\end{center}
\vspace{0.2in}
\fboxsep0.5cm

In addition to volumes of revolution, we can also use integrals to compute any volume in which we can describe the cross-section perpendicular to the direction of integration.\\

EXAMPLES:\\

Find the volume of a pyramid with square base with side length $s$ and height $h$.


\vspace{2.5in}

The base of a volume is the triangular region with corners at $(0,0)$, $(1,0)$, and $(0,3)$.  Cross-sections perpendicular to the $y$-axis
are squares.  What is the volume?

\vspace{2.8in}

Another volume has the same base, but cross-sections are isosceles right triangles with one leg on the base.  What is the volume?


\pagebreak

\begin{center}
\Large
\rm{Chapter 6.3:  Method of Shells}
\\
\end{center}

The {\bf method of shells} is another way to calculate volumes of revolution by dividing them up in a different way.


\vspace{4in}


EXAMPLES:\\

\begin{enumerate}
  \item{An area is bounded by $y=0$, $y=x-x^2$.  Find the volume generated by rotating the area around the $y$-axis.}


  \pagebreak

\item{An area is bounded by $y=x^3$, $y=8$, and $x=0$.  Find the volume generated by rotating the area around the $y$-axis.\\
  (\emph{We calculated this volume with the method of disks and found it was 96$\pi$/5.}) }  

  \vspace{3.5in}

  \item{An area is bounded by $y=0$, $y=x^2$, and $x=2$.  Find the volume generated by rotating the area around the line $x=3$.}

    \vspace{3.5in}
\end{enumerate}
EXERCISE:  Find the volume obtained by rotating the area bounded by $y^2-x^2=1$ and $y=2$ around the $x$-axis.  Show that
the method of disks and the method of shells give the same answer, $4\pi\sqrt{3}$.

\pagebreak

\begin{center}
\Large
\rm{Volume, Mass, and Density}
\\
\end{center}

The volume of an object is related to its mass through density.  \\

EXAMPLES:\\

Uniform density:  A large cylinder has radius 2 m, height 3 m, and a density of 500 kg/m$^3$.  What is the mass?\\

\vspace{0.6in}

Linear density:  A wire has variable density $\rho(x) = e^{-x/100}$ g/cm, for $0\leq x\leq 100$, where $x$ is the distance from the left end of the wire.  Construct an integral to calculate the mass. \\

\vspace{3in}

A lake is in the shape of a cylinder with radius 200 m and depth 5 m.  The density of a certain species of algae is given by
$p(x) = \frac{1}{10x+50}$ kg/m$^3$.  What is the mass of all the algae in the lake?

\pagebreak

\begin{center}
\Large
\rm{Work}
\\
\end{center}
In physics the work done by applying a constant force is defined as the product of force and distance.  In common examples, the force
will be applied to overcome some resistance such as a spring, an electric field, or gravity.  The force needed to overcome gravity and lift
an object is called the weight of the object. It is calculated by $F=mg$, where $g=9.8$ m/s is a constant.\\

EXAMPLE:\\

Determine the weight of the cylinder with radius 2 m, height 3 m, and a density 500 kg/m$^3$.  Determine the work done in lifting the object up 10 m.

\vspace{2in}

NOTE:  In US units mass and weight are both named pounds (lbs).  Thus the weight of a 10 pound book is 10 pounds.  The work required to lift
a 10 pound book up 5 ft is 50 ft-lbs.\\


VARIABLE FORCE EXAMPLE:\\

Suppose the force acting on a particle at postion $x$ m is given as a function $f(x) = (3x^2+x)$ N.  What is the work required to move the object from postion $x=1$ to position $x=3$?

\pagebreak


SPRING EXAMPLES:\\

A common example of variable force is that required to stretch a spring.  In the simplest scenario the force required to stretch the spring is
proportional to the length the spring has already been stretched.  \\

\vspace{2in}

Suppose a 40 N force is required to hold a spring at 5 cm past its natural length of 10 cm.  How much work was required to stretch the spring
from its natural length to the current length?

\vspace{3in}

If the work required to stretch a spring 8 in beyond its natural length is 20 ft-lbs, what is the work required to stretch the spring 16 in beyond its natural length?

\pagebreak

CABLE EXAMPLES:\\

A cable has length 50 m and mass 8 kg/m.  We lift the cable up by one end a distance of 20 m.  How much work is done?\\


\vspace{3in}

A 400 lb chain hangs vertically 200 ft from a crane.  What is the work required to haul the chain to the top of the crane?\\

\vspace{3in}


EXERCISES:\\

Suppose the same chain is lifted only half way up.  What is the work done?\\

Suppose the same chain is lifted by pulling the bottom end up to meet the top end.  The chain ends with both ends at the top of the crane,
and a loop hanging down 100 ft.  What is the work done?\\

\pagebreak

TANK EXAMPLES:\\

A tank is in the shape of an inverted cone (the point is at the bottom) with radius 3 m, height 8 m.  It is filled with a fluid that
has density 1100 kg/m$^3$.  What is the work required to pump all the fluid to the top of the tank?\\

\vspace{4in}

A tank has a rectangular top which is 6 ft long and 2 ft wide.  One end of the tank is 4 ft m deep, the other end is 6 ft deep.  The tank
is filled with water so that the level of water is 1 ft from the top of the tank.  What is the work required to pump all of the water to
the top of the tank?  The density of water is 62.5 lbs/ft$^3$.\\

\vspace{4in}

EXERCISE: \\

A rope of mass 0.8 kg/m is used to lift a leaky 10 kg bucket from the bottom of a well up to ground level.  The well is 12 m
deep and the bucket is lifted at a constant speed.  At the the bottom of the well the bucket contains 36 kg of water, but the water leaks
at a constant rate and empties just as the bucket reaches ground level.  How much work was done to collect no water from this well?



\end{document}


