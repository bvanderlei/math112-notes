\documentclass[11pt]{article}
\usepackage[letterpaper, margin=1in]{geometry}
\usepackage{amsmath, amssymb, graphicx, epsfig, fleqn}
\setlength{\parindent}{0pt}
\newcommand{\ud}{\,\mathrm{d}}
\everymath{\displaystyle}
\def\FillInBlank{\rule{1in}{.01in} }
\pagestyle{empty}

\begin{document}
\begin{center}
\Large
\rm{Math 112}
\\
\rm{Chapter 7.4:  Partial Fractions}
\\
\end{center}
\vspace{0.2in}

The goal of this section is to integrate rational functions (ratios of polynomials).\\

\vspace{2in.}


EXAMPLE:\\

$\frac{1}{x+2} + \frac{3}{x-1}$

\vspace{2in.}

{\bf Partial fractions} is an algebraic method that reverses the addition of rational functions.\\

EXAMPLE:\\

$\int \frac{4x-3}{x^2+3x} \, dx$


\pagebreak

Partial fractions works in general by proposing a possible decomposition with unknown constants, and then applying algebra to
solve for the constants.  We can break down the study of partial fractions into cases that depend on how the denominator factors.\\

CASE I: Denominator factors into a number of distinct linear factors.\\

EXAMPLES:\\

$\int \frac{2x+6}{x^2-5x+6} \, dx$

\vspace{3in}

$\int \frac{x^2+1}{x^3-2x^2-3x} \, dx$

\vspace{2in}

\pagebreak

CASE II: Denominator factors into a number of linear factors that are not all distinct.\\

EXAMPLES:\\

$\int \frac{x-1}{x^2(x+2)} \, dx$

\vspace{4in}

$\int \frac{1}{(x^2-1)^2} \, dx$


\pagebreak

CASE III: Denominator factors into linear factors and distinct irreducible quadratic factors.\\

EXAMPLES:\\

$\int \frac{1}{(x-1)(x^2+5)} \, dx$

\vspace{3.5in}

$\int \frac{x-1}{(x+2)(3x^2+1)} \, dx$


\pagebreak

$\int \frac{4x^2-x+1}{(x^2+2x+2)(x-3)} \, dx$



\vspace{6in}


In general the simplest integrals we will get in these three cases have the following forms:\\

$\int \frac{A}{x-a} \, dx$ \hspace{1in} $\int \frac{A}{(x-a)^n} \, dx$ \hspace{1in} $\int \frac{Ax+B}{ax^2+bx+c} \, dx$


\pagebreak

EXAMPLES: IMPROPER RATIONAL FUNCTIONS\\

$\int \frac{2x^3+x}{x^2-x} \, dx$

\vspace{3.5in}


$\int \frac{x^3+4x^2+x-1}{x^3+x^2} \, dx$
%% \pagebreak

%% The steps to follow to evaluate integrals of rational functions:

%% \begin{enumerate}
%% 	\item {Do long division first if degree of numerator is greater or equal to degree of denominator.}
%% 	\item{Factor denominator if possible and decide which case applies.}
%% 	\item{Carry out algebra to find the unknown constants.}
%% 	\item{Evaluate the integrals of the simpler functions.}
%% \end{enumerate}


\end{document}


