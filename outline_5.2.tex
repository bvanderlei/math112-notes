\documentclass[11pt]{article}
\usepackage[letterpaper, margin=1in]{geometry}
\usepackage{amsmath, amssymb, graphicx, epsfig, fleqn}
\setlength{\parindent}{0pt}
\newcommand{\ud}{\,\mathrm{d}}
\everymath{\displaystyle}
\def\FillInBlank{\rule{2.5in}{.01in} }
\pagestyle{empty}

\begin{document}
\begin{center}
\Large
\rm{Math 112}
\\
\rm{Chapter 5.2:  The Definite Integral}
\\
\end{center}
\vspace{0.2in}
\fboxsep0.5cm

For a generic area defined by $a \leq x \leq b$, and $0 \leq y \leq f(x)$  where $f$ is a positive function, we can make
an estimate using a {\bf Riemann Sum}.



\vspace{2.6in}


The limit of the approximation as $N\to\infty$ is what we call the {\bf definite integral}.\\

\vspace{2.5in}


NOTES:\\

\vspace{1.5in}

\pagebreak

We adopt some \emph{conventions} so that the integral definition makes sense in other situations.\\


\vspace{4.5in}

EXAMPLES:\\

$\int_0^3 x-1\,dx$

\vspace{2in}


$\int_1^0 \sqrt{1-x^2}\,dx$


\vspace{2.5in}


$\int_{-1}^3 |x|\,dx$


\vspace{1.5in}

$\int_{0}^{2\pi} \cos{x}\,dx$

\vspace{1in}

PROPERTIES:

\vspace{4in}

EXAMPLES:\\

$\int_{0}^{1}2x+3(1-x^2)\, dx$

\vspace{1.5in}

If   $\int_{2}^{6}f(x)\, dx = 7$ and $\int_{2}^{3}f(x)\, dx = 5$, find $\int_{3}^{6}f(x)\, dx$  

\vspace{1.5in}


COMPARISON PRINCIPLES:

\vspace{4in}


EXAMPLE: \\

Give upper and lower bounds on the value of $\int_0^2e^{-x}\, dx$.


\pagebreak

We can always use Riemann sums to \emph{approximate} the value of an integral.  In addition to $R_N$ and $L_N$,
we could use other sums such as the {\bf Midpoint Rule}.  


\vspace{3in}

EXAMPLE: \\

Find the $M_4$ approximation of  $\int_0^2e^{-x}\, dx$.

\vspace{2.5in}

Approximations do better with larger $N$.

\end{document}


