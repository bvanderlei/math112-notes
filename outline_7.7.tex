\documentclass[11pt]{article}
\usepackage[letterpaper, margin=1in]{geometry}
\usepackage{amsmath, amssymb, graphicx, epsfig, fleqn}
\setlength{\parindent}{0pt}
\newcommand{\ud}{\,\mathrm{d}}
\everymath{\displaystyle}
\def\FillInBlank{\rule{1in}{.01in} }
\pagestyle{empty}

\begin{document}
\begin{center}
\Large
\rm{Math 112}
\\
\rm{Chapter 7.7: Numerical Integration.}
\\
\end{center}
\vspace{0.2in}

The goal of this section is to approximate the value of definite integrals by using finite sums.\\


EXAMPLE:\\

erf$(x) = \frac{2}{\sqrt{\pi}}\int_0^xe^{-t^2}\, dt$ \quad  Suppose we want to know erf(1).

\vspace{0.2in}

Estimate $\int_0^1e^{-x^2}\, dx$ using $L_4$, $R_4$, and $M_4$ \\

\vspace{2in}

It would be even more useful if we could bound the difference $\left|L_4-\int_0^1e^{-x^2}\, dx\right|$\\

We could then ask what size of $N$ do we need to make $L_N$ ``good enough''



\pagebreak


If we want a more accurate approximation, we might try using shapes other than rectangles.  Estimate $\int_0^1e^{-x^2}\, dx$ using 4 trapezoids.  Call this $T_4$. \\

\vspace{4in.}


{\bf Trapezoid Rule} $T_N$
\begin{displaymath}
\int_a^bf(x)\,dx \approx  \frac{\Delta x}{2}\left[f(x_0) + 2f(x_1) + 2f(x_2) + .... + 2f(x_{N-1}) + f(x_N)\right]
\end{displaymath}

\vspace{1cm}

COMPARISON OF APPROXIMATIONS \\

We can compare the different approximations by looking at the errors.  Define  the error in using the Midpoint rule as \\
\begin{displaymath}
E_M = \int_a^bf(x) \, dx - M_N
\end{displaymath}
We can define $E_L$, $E_R$, and $E_T$ similarly.

\pagebreak

To get a sense of the errors, we can approximate the value of a known integral, $\int_1^3\frac{1}{x} \, dx$.\\

Several calculations by computer gives us...\\

\vspace{0.3cm}

\begin{tabular}{|c|c|c|c|c|}\hline
$N$ & $E_L$ & $E_R$ & $E_T$ & $E_M$\\
\hline
5 & -0.1449 & 0.121678 & -0.01165 & 0.005755 \\
10 & -0.06962 & 0.063716 & -0.00295 & 0.00147 \\
20 & -0.03407 & 0.032593 & -0.00074 & 0.00037 \\
\hline
\end{tabular}
\vspace{0.2in}

We notice some trends...

\begin{enumerate}
\item{}
\item{}
\item{}
\item{}
  \end{enumerate}

\vspace{0.2in}

ERROR BOUNDS\\

For the Midpoint Rule and Trapezoid rule we have the following formula to estimate the error
\begin{displaymath}
|E_M| \leq \frac{K(b-a)^3}{24N^2} \quad\quad |E_T| \leq \frac{K(b-a)^3}{12N^2}
\end{displaymath}
where $K$ is a number such that $|f''(x)| < K$ when $x$ is in the interval $[a,b]$.

\vspace{0.3cm}

EXAMPLES:\\

Again consider the integral  $\int_1^3\frac{1}{x} \, dx$.

\begin{enumerate}
\item{If $N=20$, what is the largest that $|E_T|$ could be?  How about $|E_M|$?}
  \item{How large would $N$ need to be if we want $|E_M|< 0.0001$?}
  \end{enumerate}




\pagebreak

Estimate the error $E_T$ in the approximation $T_8$ for the following integral.
\begin{displaymath}
\int_0^1\cos{x^2} \,dx 
\end{displaymath}

\vspace{5in}

EXERCISE:
Consider the following integral:
\begin{displaymath}
\int_0^{\pi}\sin{x} \,dx 
\end{displaymath}
\begin{enumerate}
\item{Find $M_{10}$, and $T_{10}$, and the corresponding errors $E_M$, and $E_T$.}
\item{Compute the error bounds using the formulas and then compare them with the true errors $E_M$, and $E_T$.}
\item{How large do we have to choose $N$ so that $|E_T| < 0.00001$? Use the error bound formula.}
\end{enumerate}


\pagebreak

If we want to make a more accurate approximation for a fixed $N$, we might approximate area with different shapes.  {\bf Simpson's Rule} uses parabolas to approximate a function $f$, and then estimates the integral of $f$ by using integrals of quadratic functions.

\vspace{6in}

{\bf Simpson's Rule}
\begin{displaymath}
\int_a^bf(x)\,dx \approx S_N = \frac{\Delta x}{3}\left[f(x_0) + 4f(x_1) + 2f(x_2) +  4f(x_3) + .... + 2f(x_{N-2}) + 4f(x_{N-1}) + f(x_N)\right]
\end{displaymath}
where $N$ {\bf is an even integer.}

\pagebreak

For Simpson's Rule we have a similar formula to estimate the error
\begin{displaymath}
|E_S| \leq \frac{K(b-a)^5}{180N^4} 
\end{displaymath}
where $K$ is a number such that $|f''''(x)| < K$ when $x$ is in the interval $[a,b]$.

\vspace{0.5cm}

EXAMPLE:\\

For the integral $\int_1^3 \frac{1}{x} \, dx$  how large do we have to choose $N$ so that $|E_S| < 0.0001$? 

\vspace{5in}
EXERCISE:
Consider the following integral:
\begin{displaymath}
\int_0^{\pi}\sin{x} \,dx 
\end{displaymath}
\begin{enumerate}
\item{Find $S_{10}$ and the corresponding error $E_S$.}
\item{Compute the error bound using the formula and then compare them with the true error $E_S$.}
\item{How large do we have to choose $N$ so that $|E_S| < 0.00001$?  Use the error bound formula.}
\end{enumerate}

\end{document}


