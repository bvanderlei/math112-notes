\documentclass[11pt]{article}
\usepackage[letterpaper, margin=1in]{geometry}
\usepackage{amsmath, amssymb, graphicx, epsfig, fleqn}
\setlength{\parindent}{0pt}
\newcommand{\ud}{\,\mathrm{d}}
\everymath{\displaystyle}
\def\FillInBlank{\rule{2.5in}{.01in} }
\pagestyle{empty}

\begin{document}
\begin{center}
\Large
\rm{Math 112}
\\
\rm{Chapter 11.10:  Taylor Series}
\\
\end{center}
\vspace{0.2in}
\fboxsep0.5cm

THEOREM: \\

If $f$ has a power series representation at $a$, 

\begin{displaymath}
f(x) =\sum_{n=0}^{\infty}c_n(x-a)^n  \hspace{0.7in} |x-a|<R
  \end{displaymath}
then the coefficients are given by the formula
\begin{displaymath}
c_n=\frac{f^{(n)}(a)}{n!}
  \end{displaymath}

\vspace{0.2in}

EXAMPLE:\\

$f(x) = e^x$ and $a=0$

\vspace{3in}

When we write the series with this formula, we call it a  {\bf Taylor series}.\\

\begin{displaymath}
f(x) = f(a) + \frac{f'(a)}{1!}(x-a) + \frac{f''(a)}{2!}(x-a)^2 + \frac{f'''(a)}{3!}(x-a)^3 + .... 
  \end{displaymath}

and when $a=0$ we call it a {\bf Maclaurin series}.\\

\begin{displaymath}
f(x) = f(0) + \frac{f'(0)}{1!}x + \frac{f''(0)}{2!}x^2 + \frac{f'''(0)}{3!}x^3 + .... 
  \end{displaymath}


\pagebreak

When we use only a finite number of terms, we call the sum a {\bf Taylor polynomial}.\\

\vspace{3in}


We can use the Maclaurin series for $e^x$ to construct series of related functions.\\


EXAMPLES:\\

Find the Maclaurin series for the following functions.\\

$g(x) = e^{-x}$

\vspace{1.5in}

$h(x) = x^2e^{2x}$

\vspace{1.5in}

$f(x) = e^{-(x-2)}$

\pagebreak
We can make different Taylor series for the same function \emph{centered} at different $a$.\\

EXAMPLE:\\

$f(x) = e^x$ at $a=4$.

 \vspace{4.5in}


 
 THEOREM: \\

 Let $R_N = f(x) - T_N(x)$, and let $M$ be number such that $|f^{n+1}(x)|\leq M$ for $|x-a|\leq d$.
 Then the remainder $R_N(x)$ satisfies

 \begin{displaymath}
|R_N(x)| \leq \frac{M}{(N+1)!}|x-a|^{N+1} \hspace{1in} \mbox{for} |x-a|<d
    \end{displaymath}

 \pagebreak

 EXAMPLE:\\
 
 $f(x) = e^x$ and $a=0$

\vspace{4in}

 MORE EXAMPLES:\\

 Find the Maclaurin series for $\sin{x}$.

 \pagebreak

  Find the Maclaurin series for $\cos{x}$.

  \vspace{4in}

  Find the Taylor series for $\cos{x}$ at $a=\pi/2$.

\pagebreak
  
  Find the Taylor series for $2^{x}$ at $a=3$.\\

  \vspace{4in}
  


  
  Find the Taylor series for $x^3+2x-1$ at $a=1$.

\pagebreak


  Find the Maclaurin series for $\ln(x+1)$. \\
   (\emph{Note we've found this series already using the geometric series.})

 \vspace{3.5in}

As before, we can find other series by manipulating series we know.\\

Find the first few terms in the Maclaurin series for $e^x\sin{x}$.\\

 \vspace{2in}

Find the first few terms in the Maclaurin series for $\tan{x}$.\\

%\pagebreak
%
%  APPLICATIONS:  (Integrals and Erf, approximations (sqrt and ln), de, limits?)
%  
%Find $T_3(x)$ for $\sqrt{x}$ at $a=4$ and use it to estimate $\sqrt{4.3}$ Estimate the error in the approximation.
%
%
%\vspace{1in}
% \pagebreak

\end{document}




