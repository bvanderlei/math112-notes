\documentclass[11pt]{article}
\usepackage[letterpaper, margin=1in]{geometry}
\usepackage{amsmath, amssymb, graphicx, epsfig, fleqn}
\setlength{\parindent}{0pt}
\newcommand{\ud}{\,\mathrm{d}}
\everymath{\displaystyle}
\def\FillInBlank{\rule{2.5in}{.01in} }
\pagestyle{empty}

\begin{document}
\begin{center}
\Large
\rm{Math 112}
\\
\rm{Chapter 9.5:  Linear Differential Equations}\\

\end{center}
\vspace{0.2in}
\fboxsep0.5cm

\vspace{0.2in}

A differential equation is called {\bf linear} if can be written in the form

\begin{displaymath}
  \frac{dy}{dx} + P(x)y = Q(x)
\end{displaymath}

\vspace{0.2in}
where $P$ and $Q$ are continuous functions on some interval.  

\vspace{.5in}

Solution method is based on the Product Rule

\begin{displaymath}
  \frac{d}{dx}[xy] = y + x \frac{dy}{dx}
\end{displaymath}

\vspace{0.2in}
  
  EXAMPLE:\\

\begin{displaymath}
x \frac{dy}{dx} + y = x
\end{displaymath}
  
\vspace{1.5in}


More generally, we will need to multiply by an {\bf integrating factor} $u(x)$ in order to use the Product Rule.

\begin{displaymath}
y' -2y = 3e^x
\end{displaymath}

\pagebreak

Solve the initial value problem:\\

\begin{displaymath}
  \left\{ \begin{array}{ll}
  xy'+2y = 4x^2 \\
y(1) = 2 \\
\end{array} \right.
\end{displaymath}

\vspace{3.5in}

We can write down the formula for a solution in terms of $P$ and $Q$

\begin{displaymath}
  \frac{dy}{dx} + P(x)y = Q(x)
\end{displaymath}


\vspace{2.5in}

\pagebreak


APPLICATIONS:\\

A tank contains 1000 L of whiskey 35\% alcohol.   At $t=0$, a whiskey with concentration 45\% alcohol begins
flowing into the tank at a rate of 2 L/min. If the tank is kept well-stirred and is drained at a rate of 5 L/min,
find the concentration of the whiskey blend in the tank as a function of time.

\pagebreak

The following modification of Newton's Law of Cooling can be used in the case that the ambient temperature varies in time.
    \begin{displaymath}
  \left\{ \begin{array}{ll}
\frac{dT}{dt} = k(A(t) - T)\\
T(0) = T_0 \\
\end{array} \right.
  \end{displaymath}
    Suppose for example that $A(t)$ represents the outside temperature and $T(t)$ is the interior temperature of a building with no climate control.  If we
    measure $T$ in hours, it might be reasonable to assume $A$ is a periodic function such as $A(t) = 14 + 5\cos{\frac{\pi t}{12}}$.\\

    Let's take $T_0 = 19$, $k=0.15$/hr, and see what the model predicts.


\end{document}




